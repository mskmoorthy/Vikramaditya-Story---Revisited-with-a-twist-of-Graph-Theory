\chapter{Summary}
Here at journey's end, we summarize which concepts were covered in each chapter. If some of these concepts are not clear, please go through the chapter again or contact the first author, mskgraph149@gmail.com.

In the \textbf{first chapter}, we defined a \textbf{graph} in terms of \textbf{vertices}, \textbf{edges}, and \textbf{relations} between them. A few example graphs were given to illustrate the concept.

In the \textbf{second chapter}, we introduced the \textbf{main characters} of our story.

In the \textbf{third chapter}, we explained the \textbf{degree} of a vertex and provided some elementary properties of degrees.

In the \textbf{fourth chapter}, we discussed \textbf{trees} and \textbf{rooted trees}, as well as some example applications.

In the \textbf{fifth chapter}, we discussed \textbf{subgraphs} and the different types of subgraphs.

In the \textbf{sixth chapter}, we covered a specific type of \textbf{path} and a specific type of \textbf{cycle}, i.e., the \textbf{Eulerian path} and the \textbf{Eulerian cycle}.

In the \textbf{seventh chapter}, we introduced another type of path and cycle, i.e., the \textbf{Hamiltonian path} and the \textbf{Hamiltonian cycle}.

In the \textbf{eighth chapter}. we discussed \textbf{isomorphism} and \textbf{subgraph isomorphism} between graphs.

In the \textbf{ninth chapter}, we covered the embedding of graphs in a plane, i.e.,~\textbf{planar graphs}.

In the \textbf{tenth chapter}, we introduced various \textbf{graph coloring problems} and showed a few examples.

In the \textbf{eleventh and twelfth chapters}, we discussed the two closely related problems of \textbf{spanning trees} and \textbf{shortest paths}.
In these chapters, we also provided methods to compute minimum cost spanning trees and shortest paths.

In the \textbf{thirteen, fourteenth chapters}, we described special types of subgraphs, including \textbf{cliques}, \textbf{independent sets}, \textbf{vertex covers}, \textbf{bridges}, \textbf{cut vertices}, and \textbf{cut sets}.
We also discussed applications of these special subgraphs.

In the \textbf{fifteenth chapter}, we covered \textbf{connected} and \textbf{strongly connected directed graphs}, including a football matches example.

In the \textbf{sixteenth chapter}, we discussed both \textbf{matching problems} and \textbf{assignment problems}, including the \textbf{stable marriage problem}.

In the \textbf{seventeenth chapter}, we discussed various \textbf{graph operations}, including the \textbf{graph union}, the \textbf{graph complement}, \textbf{vertex addition}, the \textbf{Cartesian product}, and the \textbf{line graph}.

In the \textbf{eighteenth chapter}, we introduced the \textbf{dominating set} of a graph.

In the \textbf{nineteenth chapter}, we covered the \textbf{graceful numbering} of a graph.

In the \textbf{twentieth chapter}, we discussed \textbf{brute force searching} techniques, including \textbf{depth first search} and \textbf{breadth first search}.

In the \textbf{twenty-one chapter}, we discussed \textbf{social networks} as an application of graphs.

Finally, in the \textbf{twenty-second chapter}, we presented the \textbf{instant insanity puzzle} and how graph theoretical techniques can be used to solve it.
