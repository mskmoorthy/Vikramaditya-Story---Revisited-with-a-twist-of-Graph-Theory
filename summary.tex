\chapter{Summary}
Here we summarize what concept is covered in each chapter. If some of these concepts are not clear, please browse through the chapter again or contact the first author mskgraph149@gmail.com

In the \textbf{second chapter} we have given the definition of a \textbf{graph+}in terms of \textbf{vertices}, \textbf{edges} and \textbf{relations} between them. A few examples of graphs are given to illustrate the concept. 

In \textbf{third chapter}, the \textbf{main characters} in this story are introduced.

In the \textbf{fourth chapter}, \textbf{degree of} a vertex is explained and some elementary properties of degrees are given.

In the \textbf{fifth chapter}, we discuss \textbf{trees} and \textbf{rooted trees} and some example applications.

In the \textbf{sixth chapter}, we discuss \textbf{subgraphs} and different types of subgraphs.

In the \textbf{seventh chapter} a type of path, and cycle, namely, \textbf{Eulerian paths and cycles} is discussed.

In the \textbf{eighth chapter} another type of path and cycle, namely, 
\textbf{Hamiltionian paths and cycles} is discussed.

In the \textbf{ninth chapter}. we discuss \textbf{ isomorphism} and \textbf{subgraph isomorphism} between graphs.

In the \textbf{tenth chapter}, embedding of graphs in a plane , i.e. \textbf{planar graphs} is discussed.

In the \textbf{eleventh chapter}, we introduce various \textbf{coloring problems} and show a few examples.

In the \textbf{twelfth and thirteenth chapters}, we discuss two closely related problems of \textbf{spanning trees} and \textbf{shortest paths} problems. In this chapter, we also provide methods to compute minimum cost spanning tree and shortest path.

In the \textbf{fourteenth, fifteenth chapters}, we describe special kinds of subgraphs , like \textbf{cliques, independent sets, vertex covers, brdges, cut vertices} and \textbf{ cut sets}. We also discuss applications of  the subgraphs. We discuss \textbf{matching, assignment problem} in the \textbf{sixteenth chapter}. The. \textbf{Stable Marriage Problem} was also discussed in this chapter.

\textbf{Graph Operations} is discussed in \textbf{chapter eighteen}. \textbf{Dominating set} and \textbf{Graceful Numbering} is discussed in \textbf{chapter nineteen} and \textbf{chapter twenty}

\textbf{Brute force searching} and \textbf{social networks} are discussed in \textbf{chapters twenty one} and \textbf{twenty two}

In \textbf{chapter twenty three} we discuss the \textbf{Instant insanity} puzzle and how Graph Theoretical 
technique is used to solve it.







